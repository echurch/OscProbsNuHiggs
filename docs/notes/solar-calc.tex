
%\RequirePackage{lineno}
\documentclass[reprint,onecolumn,superscriptaddress,preprintnumbers]{revtex4}%
\usepackage[a4paper]{geometry}
%\usepackage{draftwatermark}
%\SetWatermarkScale{5}
%\SetWatermarkLightness{0.9}
\usepackage{graphicx}
%\usepackage{subfig}
\usepackage{bm}
\usepackage{latexsym}
\usepackage{epsf}
\usepackage{rotating}
\usepackage{epsfig,graphics,rotate,xcolor}
\usepackage{wrapfig}
\usepackage{amssymb}
\usepackage{amsmath}
\usepackage{amsfonts}
%%\usepackage{subfigure}
\usepackage{array,hhline,dcolumn}%
\usepackage[normalem]{ulem}
\usepackage{url}
\usepackage{verbatim}

%\usepackage[justification=justified,singlelinecheck=false]{caption}
\setcounter{MaxMatrixCols}{30}
%TCIDATA{OutputFilter=latex2.dll}
%TCIDATA{Version=5.50.0.2953}
%TCIDATA{LastRevised=Monday, December 14, 2009 10:40:56}
%TCIDATA{<META NAME="GraphicsSave" CONTENT="32">}
%TCIDATA{<META NAME="SaveForMode" CONTENT="1">}
%TCIDATA{BibliographyScheme=Manual}
%TCIDATA{Language=American English}
%BeginMSIPreambleData
\providecommand{\U}[1]{\protect\rule{.1in}{.1in}}
%EndMSIPreambleData
\bibliographystyle{plain}



\begin{document}

\title{Solar nu xsection on overdense CNB}

\author{Eric D.~Church}


\begin{abstract} 
Solar neutrinos  interact  on our necessarily over-dense CNB. This write-up shows the calculation of that rate. A jupyter notebook also exists in this repo to place a limit in $n$-$g$ space so as not to diminish the solar $\nu_e$ flux beyond 1\% or so of its very well-known value, which parameter space is not tenable.
\end{abstract}
\maketitle

\section{Motivation}

To calculate the interaction rate we want to appeal to the optical theorem, 
\begin{equation}
\sigma = 4\pi/k \cdot \Im (f(\theta=0)) \label{eqnot}
\end{equation}. 

That is because in our paper we have calculated matrix elements in the very forward direction and come up with some tidy expressions (eqn 8 and 9 of the paper) for the scattering amplitudes. If, instead, we need to fully calculate the xsection from scratch we'll need to go back to the 4 diagrams and sum them with general kinematics, square them and integrate, blah blah blah. It'd be non-trivial. 

\section{Intro}
See~\cite{optthm} for a very nice elaboration of the optical theorem. Basically, one writes a plane wave $\psi=\exp^{ikz}$, decomposes in legendre polynomials and tacks on extra phases for the out-going part of the wave. The differential cross-section is calculated with $\psi^*\psi$.
\begin{equation}
\frac{d\sigma}{d\Omega} = |f(\theta,\phi)|^2 \label{eqnf2}
\end{equation}
An expression is shown in that reference for $f(\theta,\phi)$ which we don't list here. But, because it contains legendre polynomials, when one calculates $\sigma_T = \int d\Omega |f(\theta,\phi)|^2$ there's a completeness/orthogonality relation that leads straight away to eqn~\ref{eqnot}.

Namely, we can hope to get at the total cross-section we need by using eqn 9 of our paper.

\section{MFT in the house}
We have the above expression~\ref{eqnf2} for the differential cross-section. The general one from Fermi's Golden Rule for 2-particles-goes-to-2-particles cross-section is
\begin{equation}
\frac{d\sigma}{d\Omega} = \frac{1}{(2\pi)^24E_1E_2(\Phi_1+\Phi_2)} |M_{fi}|^2 \frac{d^3k_3}{2E_3} \frac{d^3k_4}{2E_4} *\delta(p+k_2+k_3+k).
\end{equation}

1,2,3,4 are the incoming neutrino, CNB neutrino, outgoing neutrino, leftover CNB neutrino. The $\Phi$s are the fluxes of 1 and 2. Our boy MFT on pg37 of~\cite{mft}integrates this out for a very light incoming particle, just as in our solar $\nu_e$ beam.
\begin{equation}
\frac{d\sigma}{d\Omega} = \frac{1}{64\pi^2}(\frac{E_3}{ME_1})^2 |M_{fi}|^2
\end{equation}
The $M$  the target particle, is our CNB neutrino $m_i$. $E_3$ is our out-going neutrino k. 

\section{Extracting $f(\theta=0)$}
Let's write this as 
\begin{equation}
\frac{d\sigma}{d\Omega} = \frac{1}{64\pi^2}(\frac{E_3}{ME_1})^2[\Re(M_{fi})^2+\Im(M_{fi})^2].
\end{equation}

If we similarly express~\ref{eqnf2} as the sum of real and imaginary parts squared and recognize that the only complex phase is held in $f$ and $M_{fi}$, then we can equate
the imaginary parts up to the coefficient, and evaluate at $\theta=0$
\begin{equation}
\Im (f(\theta=0)) = \frac{1}{8\pi}\frac{E_3}{mE_1}\Im(M_{fi}(\theta=0))
\end{equation}

$E_3 = k$, $k=E_1$, $M_{fi}=[2^4E_1E_2E_3E_4]^{1/2}*T$, $E_2=m_i, E_4=m_i$

MFT's $M_{fi}$ here is unitless. $T$ is our $M$ in our paper of eqns 8 and 9, I claim, from examining MFT's pg 16-17. Please check! So, finally, we use the optical theorem from equation~\ref{eqnot} of this note and eqn9 from our paper.

\begin{equation}
\sigma=\frac{1}{2k^2} \cdot \mathrm{eqn9} \cdot [2^4E_1E_2E_3E_4]^{1/2}
\end{equation}

Where eqn9 is the expression for the imaginary part of scattering matrix $M$ from our paper. So, finally

\begin{equation}
\sigma=\frac{2m_i}{k} \cdot \mathrm{eqn9}
\end{equation}


\begin{figure}[h]
\begin{centering}
\includegraphics[width=0.49\columnwidth]{../..//solar-constraint.png}
\caption{The allowed region is below and left of the center line. The lines show the limit for requiring that our CNB only consumes (0.1,1.0,10.0) times 1\% of the solar $nu$ flux. The calculation of the diminishment of the solar neutrino flux is explained in the comments in the ipynb that makes this plot.}
\end{centering}
\end{figure}


 

\bibliographystyle{apsrev}   
%%\bibliography{clean-sensbib}
\begin{thebibliography}{9}
\bibitem {optthm} https://quantummechanics.ucsd.edu/ph130a/130\_notes/node441.html
\bibitem {mft} http://www.hep.phy.cam.ac.uk/\~thomson/partIIIparticles/handouts/Handout\_1\_2011.pdf  
\end{thebibliography}

\end{document}

